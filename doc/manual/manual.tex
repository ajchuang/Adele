\documentclass[11pt,letterpaper]{article}
\usepackage{listings}
\lstset{basicstyle=\small\ttfamily, breaklines=true}

\def\changemargin#1#2{\list{}{\rightmargin#2\leftmargin#1}\item[]}
\let\endchangemargin=\endlist 

\begin{document}
\title {Language Adele LRM}
\author {
	Jen-Chieh Huang \texttt{(jh3478)} \and Xiuhan Hu \texttt{(xh2234)} \and     
	Zixuan Gong \texttt{(zg2203)} \and Jie-Gang Kuang \texttt{(jk3735)} \and 
	Yuan Lin \texttt{(ly2324)}
}
\maketitle
\pagebreak
\begin{changemargin}{-0.5cm}{-0.5cm} 
\setcounter{tocdepth}{3}
\tableofcontents
\end{changemargin}
\pagebreak
\section {Introduction}
Adele is a programming language designed specifically for simplifying the creation of ASCII artwork. The output space is defined as a canvas object on which the user is allowed to put several custom layers. By controlling the custom layers properly, the user will be able to create interactive ASCII artwork with reusable components. To further simplify these operations, various special operator and predefined functions are included in the language specification.

In addition to the extra constructs for ASCII artwork, Adele also supports common elements such as logic control and loop control facilities. One can not only use Adele for ASCII artwork, but other more general tasks.

In this reference manual, the fundamentals of the Adele programming language will be introduced, and details of the syntactic features are also covered.

\section {Lexical Specifications}
\subsection {Comments}
The \texttt{\#} character introduces a comment, which terminates with a newline. Considering elegance and simplicity of our lexer, we do not provide multi-line comment construct. 

\subsection {Identifiers}
An identifier is a sequence of letters, digits and underscores. The first character of any identifier must be a letter or underscore. For example, \texttt {1a2\_3d} will not be considered as an identifier, but \texttt{a2\_3d} or \texttt{\_1a2\_3d} conforms to our standard. Identifiers may have any length, but only the first 31 characters are guaranteed to be considered. So any identifier longer than 31 characters is not recommended. And any identifier is case-sensitive, eg., \texttt{ab} and \texttt{Ab} is distinguishable by our compiler.

\subsection {Keywords}
The following identifiers are reserved as keywords including grammar keywords and predefined functions. The user is not supposed to use these keywords as identifier names. Otherwise, an error may occur.

\texttt {
	\begin{center}
    		\begin{tabular}{l l l}
			if 		& end 	& while \\
			return	& group	& int 	\\
			float	& void	& false	\\ 
			true		& draw	& print\_str		\\
			load		& sleep	& null			\\
    		\end{tabular}
	\end{center}
}

\subsection {Constants}
\subsubsection* {Number}
A number constant is either an integer or a float. A positive integer starts with a non-zero digit(\texttt{[1-9]}) and can be followed by any digit, or it's zero itself. A negative integer starts with a minus \texttt{-} then followed by a positive integer. Note that \texttt{012} and \texttt{-012} are not integers. A float starts with an integer, then a \texttt{.}, then followed by a sequence of any digits, eg., \texttt{0.1345}.

\subsubsection* {String}
A string literal is a string constant, which consists of a sequence of characters surrounded by double quotes as in \texttt{"..."}. A string is actually an array of characters.

\subsubsection* {Boolean}
A boolean constant is either the keyword \texttt{true} or \texttt{false}.

\subsubsection* {Character}
A character constant is a sequence of one or more characters enclosed in single quotes as in \texttt{x}. But you can also put it in a string like: \texttt{"abcx"}, where \texttt{x} is a character constant. A table of special character constant is shown below:
\begin{itemize}
	\item newline: \texttt{$\backslash$n}
	\item backslash: \texttt{$\backslash\backslash$}              
	\item horizontal tab: \texttt{$\backslash$t}         
    \item single quote: \texttt{$\backslash$'}         
    \item backspace: \texttt{$\backslash$b}             
    \item double quote: \texttt{$\backslash$"}  
\end{itemize}

\subsection {Punctuation and Separators}
A comma \texttt {,} is used to separate parameters in the parameter list of a function. A semicolon \texttt{;} is used to separate two lines. Note that the semicolon \texttt{;} should not appear at the end of the line which starts with \texttt{if}, \texttt{while}, \texttt{end} or the end of first line of a function declaration. 

\section {Types}
In the Adele programming language, static type checking is performed. Type inconsistencies will be reported as errors at compile time if the type conversion is not possible. 

\subsection{Basic types}
The basic types of Adele includes integers, floating point numbers, boolean, and void.

\subsubsection*{Integer type (int)} 
Integer type is defined as a signed decimal integer. The range is defined from \texttt{+9007199254740991} to \texttt{-9007199254740991}. When operating with floating number, the integer can be converted into a corresponding floating point number. When overflow or underflow happens, the behavior is not defined in the language specification. For integer constants, the lexical definition is \texttt{[-]?[1-9]+[0-9]* \textbar [0]}

Note that in the definition, we do not allow preceding 0s in front of the integer, and we also do not allow multiple 0s. That is, \texttt{00000} or \texttt{0001} are not legal integers. Besides, \texttt{-0} is also an invalid integer.

 \subsubsection*{Floating point number type (float)} 
Floating type is defined as a floating number conforming double-precision 64-bit format IEEE 754 values. Loss of precision can happen if a floating number is converted into an integer. For floating point constants, Note that we do not allow 00.00 in this case because the integer part of the definition is the same as the definition of integer.

\subsubsection*{Boolean type (bool)} 
Boolean type represents the boolean variables, \texttt{true} or \texttt{false}. Note that you can NOT convert boolean values to numeric values like integers. It is also not allowed to convert numeric values to boolean values. A compilation error will be given if the invalid conversion is used.

\subsubsection*{void type (void)} 
As the name suggests, the \texttt{void} type will not check the type until the compiler or the computation has to. This is particularly useful when you want to pass groups which may have several different types into a single function.

\subsection{Derived types}
For derived types, Adele supports the following types: arrays, functions, references, groups.

\subsubsection*{Arrays}
Array type in Adele is similar to the array type in the Java programming language. The user can put either a set of primitive-typed data or a set of user-defined type references. Note that all the data in the same array have to be the same type.

\subsubsection*{Functions} 
Functions in Adele are composed by the following parts.
	\begin{itemize}
		\item Return type: indicates the type which will be returned by the function.
		\item Function name: used to identify the function itself.
		\item Parameter list: a finite set of 2-element tuples (type and name), workes as input of the function.
		\item Function body: description of the computation to be carried out in the function.
		\item \texttt{end} marker: an \texttt{end} to indicate the end of the function
	\end{itemize}
	
\subsubsection*{Groups}

Group is a mechanism which enables data abstraction in Adele. Users can compose their own data abstraction by using the group keyword. Besides, Adele also defines a set of pre-defined groups for simplifying the user operations.
	\begin{itemize}
		\item user-defined: The group in Adele is just like the struct in C. It provides a mechanism for the users to group the logically related 	data into a type. However, other object-oriented features are not supported in Adele for simplification. Complicated operations in OOP can be scary. To define a group, the user has to start with the keyword, group, followed by a group name, and an \texttt{end} marker in the end.
		\item pre-defined: Pre-defined group includes canvas, graph.
	\end{itemize}

\section {Operators}
The following table summarizes the precedence and associativity of all the operators supported by Adele. Operators on the same line have the same precedence, and the rows are in order of decreasing precedence.

\begin{center}
    \begin{tabular}{| p{4cm} | p{4cm} |}		\hline
    Operators 		& Associativity \\ 		\hline \hline
    \texttt {() [] .} 			& left to right \\ 		\hline
    \texttt {*, /, \%} 		& left to right \\ 		\hline
    \texttt {+, -, \textbar}	& left to right \\ 		\hline
    \texttt {\textless,  \textless=,  \textgreater,  \textgreater=}	& left to right \\ 		\hline
    \texttt {==,  !=}			& left to right \\ 		\hline
    \texttt {// @}				& right to left \\ 		\hline
    
    \end{tabular}
\end{center}

The \texttt{()} operator is used in functions to list the parameters, and also provides the specific precedence in the arithmetic operations; \texttt{[]} is used for array subscription; the dot operator is used to access fields of groups. 

The operator \texttt{-} can serve as both the arithmetic subtraction operator and a graph operator.  Other special graph operators include \texttt{\textbar} and \texttt{// @} (details can be found in the following sections).

\section {Syntax}
\subsection {Literals}
\begin{itemize}
\item Number literals: integers and floating-point numbers;
\item Character literals: single ASCII characters surrounded by single quotes;
\item String literals: a sequence of characters surrounded by double quotes;
\item Boolean literals: \texttt{true} or \texttt{false};
\end{itemize}
\subsubsection {Identifiers}
An identifier is a valid expression, provided that it has been declared before use.

\subsubsection {Multiplicative Expressions}
Multiplicative expressions consists of expressions involving \texttt{*}, \texttt{/}, \texttt{\%}. The operands of all the three operators must have arithmetic type; the operands of \texttt{\%} must have integral type. As the convention, multiplicative operators have a higher precedence than additive operators. 

\subsubsection {Additive Expressions}
The additive operators include \texttt{+}, \texttt{-}. Rigorously speaking, \texttt{+} is the arithmetic addition operator, but \texttt{-} can serve as either the arithmetic subtraction operator or the graph operator ``attaching horizontally", which is determined by the type of its operands determines.

The operands of the arithmetic additive operators must be numbers.

\subsubsection {Graph Combining Expressions}
The graph operators are \texttt{-}, \texttt{\textbar}, and the ternary operator \texttt{// @}. These three operators does not have side effects on their operands.

The operands of the two binary graph operators must be graphs. When serving as a graph operator, \texttt{-} means  ``attaching horizontally". The operator \texttt{\textbar} means ``attaching vertically". As for the ternary operator, i.e. the overlay and locate operator, the first two operands must be graphs, and the third operand is a pair of integers in the form of \texttt{(x, y)}, indicating the coordinates.

The type of the value of graph combining expressions is \texttt{graph}. Notice that the part of a graph that exceeds the canvas will be truncated.

\subsubsection {Boolean Expressions}
Boolean expressions include expressions involving the relational operators:  \texttt {\textless},  \texttt {\textless=},  \texttt{\textgreater},  \texttt{\textgreater=};  and the equality operators: \texttt{==}, \texttt{!=}. The equality operators have lower precedence than the relational operators. The value of these expressions is \texttt{true} or \texttt{false}, of boolean type.

\subsubsection {Braced Expressions}
Expressions surrounded by parentheses are equivalent to the original expressions.

\subsubsection {Function Calls}
A function call expression consists of the function identifier, and a possibly empty, comma-separated list of parameters surrounded by parentheses. The expression is valid only if the function has been declared somewhere, either before or after the function call. Parameters of group type are passed by reference, and those of primitive types are passed by value. The value of a function call expression is the return value of that function.

\subsection {Statements}
There are five kinds of statement in Adele programs: declarations, expressions, \texttt{if} statements, \texttt{while} statements and \texttt{return} statements. Simple statements like declarations, expressions and return statements end with a semicolon. Control flow statements, if statements and while statements, are ended by an \texttt{end} keyword.

\subsubsection {Declarations}
A declaration statement assigns an identifier to a specific memory space according to its type. A primitive-typed variable declaration consists of a type, an identifier and an optional initializer:
\begin{lstlisting}[tabsize=4]
	type id;
	type id = expression;
\end{lstlisting}

For example:
\begin{lstlisting}[tabsize=4]
	int x; 		# declare an int variable x
	void v = x; # declare an void variable v, 
                # and v is referred to x;
\end{lstlisting}

A group structure variable is declared by given a defined group type identifier and an instance id:
\begin{lstlisting}[tabsize=4]
	group group-type id;
\end{lstlisting}

For example:
\begin{lstlisting}[tabsize=4]
	group point p1;
	group circle c1;
\end{lstlisting}

\subsubsection {Expressions}
An expression along with a semicolon can be a valid statement. For example:
\begin{lstlisting}[tabsize=4]
	sum = sum + x;
	flag = (a+b)<(a-b);
\end{lstlisting}

\subsubsection {return statements}
\texttt{return} statements will end a function call and returns control to the calling function with a return value:
\begin{lstlisting}[tabsize=4]
	return result;
	return;
\end{lstlisting}

For a function whose return type is \texttt{void}, we can omit the return value as above and pass a \texttt{null} value back to the calling function.

\subsubsection {if statements}
The \texttt{if} statement allows the user to control whether or not a program enters a block of code, based on whether a given condition is true. 

\begin{lstlisting}[tabsize=4]
	if (condition_expression)
   		statement1;
   		statement2;
		...
	end
\end{lstlisting}

The statements between the \texttt{if} line and the \texttt{end} marker will be executed if and only if \texttt{condition\_expression} returns a bool \texttt{true} value.

\subsubsection {while statements}
The \texttt{while} statement loops through a block of code as long as a specified condition is true.

\begin{lstlisting}[tabsize=4]
	while (condition_expression)
   		statement1;
	   	statement2;
		...
	end
\end{lstlisting}

The statements between the \texttt{while} line and the \texttt{end} marker will be executed iteratively until \texttt{condition\_expression} is false.

\section {Built-in Functions}
Besides the grammar defined above, Adele also provides some built-in functions to accomplish the job of rendering combined or moving ASCII art. Because the syntax of the built-in functions are simple and fixed, the built-in functions will take all the acceptable arguments and do not return any values (except for the load function). When the types of the arguments for the functions are incorrect, the compiler should report them as errors.

\subsection {draw}
The prototype of the draw function is simply
\begin{lstlisting}[tabsize=4]
	void draw ()
\end{lstlisting}
Since the draw function neither takes any argument not returns any value, the statement involving \texttt{draw()} should look just like the description above. When the draw function is invoked, Adele will render the target codes of printing all the graphs already placed on the canvas. The graphs not yet placed on the canvas or already removed from the canvas by the time when \texttt{draw()} is invoked will not be displayed.

\subsection {print}
The one-time string printing function should be invoked according to the prototype
\begin{lstlisting}[tabsize=4]
	void print_str (int x, int y, string str)
\end{lstlisting}
where \texttt{x} and \texttt{y} are the position of first character of the string on the x-axis and y-axis respectively, and \texttt{str} is the target string to be displayed. The string printing function will place the desired string on the bottom layer over the canvas for the next \texttt{draw()}. After a \texttt{draw()} is invoked, all the strings rendered and placed by the string printing function will be removed from the canvas.

\subsection {load}
To construct graphs more efficiently, a load function is provided by Adele to import ASCII art from existing files. The prototype of the load function is
\begin{lstlisting}[tabsize=4]
	reference_to_graph = load (string filename)
\end{lstlisting}	
The file will only be opened in text/string mode but not in binary mode. That is, only plain-text files will be valid for the load function. If the source file is not plain-text, the non-text part will be read in and could induce unexpected results. The load function is also the only built-in function that returns a value, i.e. a reference to a graph constructed from the source file.

\subsection {sleep}
To achieve better flow control, \texttt{sleep}, a function which pauses the program is provided in Adele to enable the user to control the flow in a more flexible way. The prototype of the sleep function is
\begin{lstlisting}[tabsize=4]
	void sleep(int millisecond)
\end{lstlisting}	
The argument for the sleep function is the duration in milliseconds. The sleep function does not return a value.

\pagebreak
\section {Appendix}
\subsection {Syntactical Grammar}
Note: the draft grammar is not finalized.

\begin{lstlisting}[tabsize=2]
grammar adele;

prog    :  /* empty programs*/ 
        |   ( 
                func       /* functions*/
        |       type_declaration /* user defined types*/
        |       (declaration SEMICOLON) /* declarations*/
            )*              
        ;
        
type_declaration:
        GROUP ID 
        (TYPE ID SEMICOLON)* 
        END
        ;

func:   (TYPE | GROUP ID) ID LPAREN plist RPAREN 
        stmts
        END ;

plist:  
        |   ( (TYPE ID COMMA) | (GROUP ID ID) )* (TYPE ID | GROUP ID ID) 
        ;

stmts:  stmt* ;
stmt:       SEMICOLON       
        |   if_stmt       
        |   while_stmt      
        |   expr SEMICOLON  
        |   declaration SEMICOLON
        |   RETURN expr SEMICOLON
        ;

if_stmt:        IF LPAREN expr RPAREN 
                stmts 
                END ;

while_stmt:     WHILE LPAREN expr RPAREN 
                stmts 
                END ;

declaration:   GROUP ID ID
            |   TYPE ID 
            |   TYPE ID EQUAL expr
            ;

expr:       LPAREN expr RPAREN  /* parenthesis */
        |   ID LPAREN func_plist RPAREN                       /* function call */
        |   expr    ADDITIVE_OP   expr        /* addition */
        |   expr    MULTIPLICATIVE_OP   expr        /* multiplication & division */
        |   expr    NE      expr        /* not equal */
        |   expr    GT      expr        /* less than */
        |   expr    LT      expr        /* less than */
        |   expr    GET     expr        /* less than */
        |   expr    LET     expr        /* less than */
        |   ID OVERLAY ID AT LPAREN NUM COMMA NUM RPAREN /* overlay */
        |   ID      EQUAL   expr        /* assignment */
        |   ID  
        |   NUM
        ; 

func_plist:  | ( fpitem COMMA )* fpitem;
fpitem:     ID | NUM | STR ;

/* keywords */
IF:     `if'        ;
END:    `end'       ;
WHILE:  `while'     ;
RETURN: `return'    ;
GROUP:  `group'     ;

/* symbols */
fragment ADD:        `+'  ;
fragment SUB:        `-'  ;
fragment MULTI:      `*'  ;
fragment DIV:        `/'  ;
LPAREN:     `('  ;
RPAREN:     `)'  ;
COMMA:      `,'  ;
SEMICOLON:  `;'  ;
EQUAL:      `='  ;
OVERLAY:    `//' ;
AT:         `@'  ;
GT:         `>'  ;
LT:         `<'  ;
GET:        `>=' ;
LET:        `<=' ;
NE:         `!=' ;

ADDITIVE_OP:    ADD | SUB ;
MULTIPLICATIVE_OP:  MULTI | DIV ;

/* types */
fragment INT:       `int'   ;
fragment FLOAT:     `float' ;
fragment CHAR:      `char'  ;
fragment VOID:      `void'  ;
fragment BOOL:      `bool'  ;
fragment STRING:    `string';
TYPE:   INT | FLOAT | CHAR | BOOL | VOID | STRING;

/* identifiers */
ID:     [_a-zA-Z]+[_0-9a-zA-Z]* ;   

/* primitive types */
fragment INT_NUM:    [-]?[1-9]+[0-9]* | [0] ;    
fragment FLOAT_NUM: INT_NUM `.' [0-9]+; 
fragment CHR:        [A-Za-z0-9_] ;
NUM:    INT_NUM ;
STR:    `"' CHR* `"' ;

WS:     [ \t\r\n]+ -> skip ;

COMMENT: `#' .* ;
\end{lstlisting}	

\end{document}

