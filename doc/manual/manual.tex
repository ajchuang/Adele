\documentclass[11pt,letterpaper]{article}
\usepackage{listings}
\begin{document}
\title {The Adele Programming Language \\ Reference Manual}
\author {
	Jen-Chieh Huang \texttt{(jh3478)} \and Xiuhan Hu \texttt{(xh2234)} \and     
	Zixuan Gong \texttt{(zg2203)} \and Jie-Gang Kuang \texttt{(jk3735)} \and 
	Yuan Lin \texttt{(ly2324)}
}
\maketitle

\section {Introduction}

Adele is a programming language designed specifically for simplifying the creation of ASCII artwork. The output space is defined as a canvas object on which the user is allowed to put several custom layers. By controlling the custom layers properly, the user will be able to create interactive ASCII artwork with reusable components. To further simplify these operations, various special operator and predefined functions are included in the language specification.

In addition to the extra constructs for ASCII artwork, Adele also supports common elements such as logic control and loop control facilities. One can not only use Adele for ASCII artwork, but other more general tasks.

In this reference manual, the fundamentals of the Adele programming language will be introduced, and details of the syntactic features are also covered.


\lstinputlisting[language=C]{../test.c}

\end{document}

